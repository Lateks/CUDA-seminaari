\documentclass[a4paper,11pt]{article}
\usepackage[utf8]{inputenc}
\usepackage[T1]{fontenc}
\usepackage[british, finnish]{babel}
\usepackage{lmodern}
\usepackage{url}
\usepackage{color}
\usepackage{graphicx}
\usepackage{setspace}
\usepackage{biblatex}
\bibliography{lahteet}

\begin{document}
\title{Järjestämisalgoritmien suunnittelu grafiikkayksiköille}
\author{Laura Leppänen}
\date{\today}
\maketitle

\tableofcontents
\onehalfspacing

\newpage

\section{Johdanto}

Järjestäminen on yksi tunnetuimmista ja tutkituimmista laskennallisista ongelmista. Järjestämisalgoritmit ovat keskeisiä rakennuspalikoita esimerkiksi tietokantajärjestelmissä, hakumoottoreissa sekä tietokonegrafiikassa ja maantieteellistä dataa käsittelevissä järjestelmissä, joissa kohteita on järjestettävä spatiaalisesti. Tästä johtuen tehokkaita järjestämisalgoritmeja tarvitaan kaikilla ohjelmointialustoilla, ja uusien arkkitehtuurien tarjoamia rinnakkaistamismahdollisuuksia on tarpeen käyttää hyödyksi.

Prosessorien ytimien määrä on jo pitkään ollut jatkuvassa kasvussa. Neliytimiset CPU-suorittimet ovat jo nykyään tavallisia, ja tulevaisuudessa kehitys näyttää menevän yhä vahvemmin ns. massiivimoniytimisten prosessorien (\emph{\foreignlanguage{english}{manycore processor}}) suuntaan. Lisäksi ytimet kykenevät usein useamman säikeen suorittamiseen yhtäaikaisesti. Grafiikkayksiköissä massiivinen ytimien määrä on jo saavutettu: esimerkiksi nykyisissä NVIDIAn valmistamissa grafiikkayksiköissä ytimiä on jopa 448 kappaletta (NVIDIA TESLA C2050/C2070/C2075) \cite{nvidiafermi2010}.

Ytimien määrän lisäksi grafiikkayksiköiden ohjelmoinnissa täytyy ottaa huomioon, että ulkoiseen muistiin viittaaminen on hidasta, ja grafiikkapiirillä olevaa nopeaa muistia on rajallinen määrä \cite{leischner2010}.

Esittelen seuraavassa luvussa lyhyesti CUDA-ohjelmointiympäristön ja ne grafiikkayksiköiden arkkitehtuuriset ominaisuudet, jotka algoritmien suunnittelussa on otettava huomioon. Sen jälkeen esittelen muutamia erityyppisiä järjestämisalgoritmeja ja niiden toteutuksia CUDA-ympäristössä.

\section{Rinnakkaislaskenta nykyisillä grafiikkayksiköillä}

% Selitä CUDA-arkkitehtuuri lyhyesti
% Tehokkuuskysymykset
% Mitä yleisiä tekijöitä täytyy ottaa huomioon algoritmien suunnittelussa

\section{Järjestämisalgoritmeja CUDA-ympäristölle}

% Tänne aliluku jokaiselle eri tyyppiselle toteutusratkaisulle
% Alkuun yleiskatsausluku?
% - radix sort (Satish et al.)
% - mergesort (Satish et al)
% - bitonic sort (Peters ja Hildebrandt)
% - count sort (Kolonias et al.)
% - sample sort (Leischner et at.)
% - quicksort (Cederman & Tsigas) -> tälle ei ehkä omaa lukuaan, turhan monimutkaista?
% Järjestämisverkot omana alilukunaan?

\printbibliography

\end{document}
